%% Template for Master thesis
%% ===========================
%%
%% You need at least KomaScript v3.0.0,
%% e.g. available in Texlive 2009
\documentclass  [
  paper    = a4,
  BCOR     = 10mm,
  twoside,
  fontsize = 12pt,
  fleqn,
  toc      = bibnumbered,
  toc      = listofnumbered,
  numbers  = noendperiod,
  headings = normal,
  listof   = leveldown,
  version  = 3.03
]                                       {scrreprt}

% used pagages
\usepackage     [utf8]                  {inputenc}
\usepackage     [T1]                    {fontenc}
\usepackage                             {color}
\usepackage                             {amsmath}
\usepackage                             {graphicx}
\usepackage     [english]               {babel}
\usepackage                             {natbib}
\usepackage                             {hyperref}

% links
\definecolor{darkblue}{rgb}{0.0,0.0,0.4}
\definecolor{darkgreen}{rgb}{0.0,0.4,0.0}
\hypersetup{
    colorlinks,
    linkcolor=black,
    citecolor=darkgreen,
    urlcolor=darkblue
}

\begin{document}
  %% title pages similar to providet template instead of maketitle
  %% Titelseiten ähnlich zum Layout des Formulars von der
%% Fakultät für Physik und Astronomie
%%
%% Weitere Infos:
%% http://www.physik.uni-heidelberg.de/aktuelles/studium/
%% (PDF link: ...studium/download/145/Vorlage_Diplomarbeit_Formular.pdf)

%% Titelintro
\thispagestyle{empty}
\begin{center}
  \renewcommand{\baselinestretch}{2.00}
  \Large\sffamily
  Fakult\"{a}t f\"{u}r Physik und Astronomie\\
  \large
  Ruprecht-Karls-Universit\"{a}t Heidelberg
  \par\vfill\normalfont
  Masterarbeit\\
  Im Studiengang Physik\\
  vorgelegt von\\
  (Vor- und Zuname)\\
  geboren in (Geburtsort)\\
  (Jahr der Abgabe)\\
\end{center}
\newpage

%% Titelseite
\thispagestyle{empty}
\begin{center}
  \renewcommand{\baselinestretch}{2.00}
  \Large\bfseries\sffamily
    (Titel)\\
    (der)\\
    (Masterarbeit)
  \par
  \vfill
  \large\normalfont
  Die Masterarbeit wurde von (Vorname Name)\\
  ausgef\"{u}hrt am\\
  (Institut)\\
  unter der Betreuung von\\
  (Frau/Herrn Prof./Priv.-Doz. Vorname Name)
  %% Bei externen Masterarbeiten hier noch den zweiten Betreuer einfügen
  %% und den vspace in Z. 45 entsprechend reduzieren
\end{center}\par
\vspace{5\baselineskip}

% Zeilenabstand zurücksetzen
\renewcommand{\baselinestretch}{1.00}\normalsize % select either german
  %% this will generate title pages similar to the template provided
%% by the Department of Physics and Astronomy Heidelberg
%%
%% More information:
%% http://www.physik.uni-heidelberg.de/aktuelles/studium/
%% (PDF link: ...studium/download/145/Vorlage_Diplomarbeit_Formular.pdf)

%% Titleintro
\thispagestyle{empty}
\begin{center}
  \renewcommand{\baselinestretch}{2.00}
  \Large\sffamily
  Department of Physics and Astronomy\\
  \large University of Heidelberg
  \par\vfill\normalfont
  Master thesis\\
  in Physics\\
  submitted by\\
  (name and surname)\\
  born in (place of birth)\\
  (year of submission)
\end{center}
\newpage

%% Titlepage
\thispagestyle{empty}
\begin{center}
  \renewcommand{\baselinestretch}{2.00}
  \Large\bfseries\sffamily
    (Title)\\
    (of)\\
    (Master thesis)
  \par
  \vfill
  \large\normalfont
  This Master thesis has been carried out by (Name Surname)\\
  at the\\
  (institute)\\
  under the supervision of\\
  (Frau/Herrn Prof./Priv.-Doz. Name Surname)
  %% additionally insert second supervisor here if carrying out an
  %% external diploma thesis. Reduce vspace in L. 44 accordingly.
\end{center}\par
\vspace{5\baselineskip}

% reset baselinestretch
\renewcommand{\baselinestretch}{1.00}\normalsize % or english title page
  %% Abstract page
%% =============
%%
%% Content of abstract pages has been put into seperate pages to simplify
%% word counting. Use e.g. the unix command
%%   wc abstract-ger.tex
%% or
%%   wc abstract-eng.tex
%% to get the number of words contained in these files.
\thispagestyle{empty}
\begin{center}
  \begin{minipage}[c][0.48\textheight][b]{0.9\textwidth}
    \small
    \textbf{
      (Titel der Masterarbeit - deutsch):
    }\par
    \vspace{\baselineskip}
    %% Latex markup und Zitate funktionieren auch hier
(Abstract in Deutsch, max. 200 Worte. Beispiel: \cite{loremIpsum})

Lorem ipsum dolor sit amet, consectetur adipisici elit, sed eiusmod tempor
incidunt ut labore et dolore magna aliqua. Ut enim ad minim veniam, quis
nostrud exercitation ullamco laboris nisi ut aliquid ex ea commodi consequat.
Quis aute iure reprehenderit in voluptate velit esse cillum dolore eu fugiat
nulla pariatur. Excepteur sint obcaecat cupiditat non proident, sunt in culpa
qui officia deserunt mollit anim id est laborum.

Duis autem vel eum iriure dolor in hendrerit in vulputate velit esse molestie
consequat, vel illum dolore eu feugiat nulla facilisis at vero eros et
accumsan et iusto odio dignissim qui blandit praesent luptatum zzril delenit
augue duis dolore te feugait nulla facilisi. Lorem ipsum dolor sit amet,
consectetuer adipiscing elit, sed diam nonummy nibh euismod tincidunt ut
laoreet dolore magna aliquam erat volutpat.

Ut wisi enim ad minim veniam, quis nostrud exerci tation ullamcorper suscipit
lobortis nisl ut aliquip ex ea commodo consequat. Duis autem vel eum iriure
dolor in hendrerit in vulputate velit esse molestie consequat, vel illum dolore
eu feugiat nulla facilisis at vero eros et accumsan et iusto odio dignissim qui
blandit praesent luptatum zzril delenit augue duis dolore te feugait nulla
facilisi.
  \end{minipage}\par
  \vfill
  \begin{minipage}[c][0.48\textheight][b]{0.9\textwidth}
    \small
    \textbf{
      (Title of Master thesis - english):
    }\par
    \vspace{\baselineskip}
    %% Latex markup and citations may be used here
(abstract in english, at most 200 words. Example: \cite{loremIpsum})

Lorem ipsum dolor sit amet, consectetur adipisici elit, sed eiusmod tempor
incidunt ut labore et dolore magna aliqua. Ut enim ad minim veniam, quis
nostrud exercitation ullamco laboris nisi ut aliquid ex ea commodi consequat.
Quis aute iure reprehenderit in voluptate velit esse cillum dolore eu fugiat
nulla pariatur. Excepteur sint obcaecat cupiditat non proident, sunt in culpa
qui officia deserunt mollit anim id est laborum.

Duis autem vel eum iriure dolor in hendrerit in vulputate velit esse molestie
consequat, vel illum dolore eu feugiat nulla facilisis at vero eros et
accumsan et iusto odio dignissim qui blandit praesent luptatum zzril delenit
augue duis dolore te feugait nulla facilisi. Lorem ipsum dolor sit amet,
consectetuer adipiscing elit, sed diam nonummy nibh euismod tincidunt ut
laoreet dolore magna aliquam erat volutpat.

Ut wisi enim ad minim veniam, quis nostrud exerci tation ullamcorper suscipit
lobortis nisl ut aliquip ex ea commodo consequat. Duis autem vel eum iriure
dolor in hendrerit in vulputate velit esse molestie consequat, vel illum dolore
eu feugiat nulla facilisis at vero eros et accumsan et iusto odio dignissim qui
blandit praesent luptatum zzril delenit augue duis dolore te feugait nulla
facilisi.
  \end{minipage}
\end{center}


  \tableofcontents
  %% Put your contents here
\chapter{Theory}
\section{Depth from focus}
\label{sec:theo depth}
One advantage of using lightfields for depth measure is its ability to get a two-dimensional image of the scene at any depth. Integrating the views of the light field camera array has the same effect as the integration of a focussed lense camera, as the lense is simply integrating slightly different viewpoints of the same scene point when focussed on the correct depth. \\
 Obtaining the refocussed integrated image is a synthetic process that only requires shifting the view coordinates artificially. Given a full four-dimensional light field $L(u, v, x, y)$ we can refocus the light field as described in \cite{ng2005}:\begin{equation}\label{eq:refocus}
L'(u, v, x, y) = L(u(1-d'), v(1-d'), x, y),
\end{equation}
where $d'$ describes the relative pixel shift. The disparity is directly related to the absolute depth of the focus (relate to PICTURE) if the relevant camera parameters are  known. Given the baseline $b$ in meters and the focal length $f$ in pixels, the depth $Z$ is given as \begin{equation}\label{key}
Z = \frac{f\cdot b}{d}.
\end{equation} 
We obtain
\begin{equation}\label{key}
\bar{L}(x,y) = \frac{1}{N_{u,v}}\int\int L'(u, v, x, y) du  dv =\frac{1}{N_{u,v}}\sum_{u}\sum_{v}  L'(u, v, x, y)
\end{equation}
Once we can focus at any range, one can adopt \textit{depth-from-focus}-techniques as described in \cite{watanabe1998rational} for depth measure. If the scene point at a given image coordinate $(x, y)$ in the center view is in focus, the contrast in the integrated image $\bar{L}(x,y)$ is high, thus a contrast measure at each pixel combined with stepwise refocussing yields a depth map. \\
For measuring the contrast, one has different options: The most straight forward approach is calculating the first derivative of the grey-value image. At high contrast structure the local intensity changes are expected to be high. Alternatively one could measure the second derivative laplacian that eventually results in higher robustness. The implementation and tests of those techniques for the benchmark dataset can be found in section \ref{label}.\\
Using a pinhole camera array allows us to go further and find a response value that shows higher consistency. Taking the absolute difference between the center view of the camera array and the refocussed image yields to promising results as shown in \cite{tao2017shape}. Under the assumption of lambertian surfaces the RGB- value of any scene point should be the same under all angles. Thus when refocussed on the correct depth, summing over all angles should result in a value that ideally is the same as in the center view alone. This is referred as \textit{photo consistency}; for more information read \cite{tao2017shape}.
The response value at a given depth is obtained from
\begin{equation}\label{key}
D'(x,y) = \frac{1}{|W_D|}\sum_{x',y' \in W_D} \left|\bar{L}(x',y')- P(x', y')\right|,
\end{equation}
where $P(x,  y)$ is the center view. For more robustness, it is averaged over a small window. We refer to this measuring technique as \textit{photo consistency} in the following. Note that calculating the absolute results in a 1-channel-image while the input images are RGB-images. \\ Tao et al. propose another measure that they refer to as $angular correspondence$. It follows the same principle, but instead of integrating the refocussed lightfield followed by comparing it to the center view, they directly take the difference of each viewpoint to the center view and sum up those differences:
\begin{equation}\label{eq:responsecorr}
D'(x,y) = \frac{1}{N_{u,v}}\sum_{u}\sum_{v}  \left|L'(u, v, x, y) - P(x,y)\right|.
\end{equation}
We tested those methods against the common contrast measures mentioned above, the results are found in section results. \chapter{results}
\section{Depth from focus}
\label{sec: depth from focus}
The depth measure using epipolar plane analysis requires iterative calculation of the structure tensor for each EPI at each disparity. A way to overcome this is to generate a preestimate of the depth before actually calculating the correct depth. This could also help to prevent possible errors due to periodic scene characteristics which can lead to mismatch errors when calculating the structure tensor. Therefore the depth pre-estimate should fulfil the following criteria:
\begin{enumerate}
	\item It should be \textit{consistent}, meaning that the number of pixels with low confidence should be the lowest possible.
	\item It should result in a \textit{fast} measure, ideally faster then it would take to do the full iterative structure tensor algorithm.
	\item It does not have to be too accurate, since it only serves as a pre-estimate. 
\end{enumerate}

The methods that are tested are described in section \ref{sec:theo depth}. We test four different ways to obtain a depth map using depth from focus:
\begin{description}
	\item[Photo consistency] This measure takes advantage of the fact that the difference between the refocussed two-dimensional image and the center view is close to zero when refocussed to  the correct depth. Response value:
	\begin{equation}\label{key}
	D'(x,y) = \frac{1}{|W_D|}\sum_{x',y' \in W_D} \left|\bar{L}(x',y')- P(x', y')\right|,
	\end{equation}
	\item[Angular correspondence] In contrast to the \textit{Photo consistency} - measure, it first calculates the absolute difference between each camera array view and the center view followed by the summation of those deviations. The response value is given as in equation \eqref{eq:responsecorr}
	\begin{equation}\label{key}
	D'(x,y) = \frac{1}{N_{u,v}}\sum_{u}\sum_{v}  \left|L'(u, v, x, y) - P(x,y)\right|
	\end{equation}
	
	\item[First derivative] The first derivative is calculated for contrast measure by applying the sobel filter onto the refocussed image $I$:
	\begin{equation}\label{key}
	 G_x=
	 \left[ {\begin{array}{ccc}
	 	-1 & 0 & 1 \\
	 	-2 & 0 & 2 \\
	 	-1 & 0 & 1 \\
	 	\end{array} } \right] \cdot I \quad G_y=
	 \left[ {\begin{array}{ccc}
	 	-1 &-2 &-1 \\
	 	0 & 0 & 0 \\
	 	1 & 0 & 1 \\
	 	\end{array} } \right] \cdot I
	\end{equation} 
	The directional gradients are simply added up to the response value
	\begin{equation}\label{key}
	D'(x,y) = |G_x(x,y)| + |G_y(x,y)|
	\end{equation}
	\item[Laplace] Here we calculate the second derivative laplacian by appling the sobel operator twice:\begin{equation}\label{key}
	D'(x,y) = \text{Laplace}(I)(x,y) = \frac{\partial^2 I}{\partial x^2}(x,y) + \frac{\partial^2 I}{\partial y^2}(x,y)
	\end{equation}
\end{description}

  \part{Appendix}
  \begin{appendix}
    \chapter{Lists}
    \listoffigures
    \listoftables
    \bibliography{references}{}
    \citestyle{egu}
    \bibliographystyle{plainnat}
    \setlength{\parindent}{0em}

Erkl\"{a}rung:\par
\vspace{3\baselineskip}
Ich versichere, dass ich diese Arbeit selbstst\"{a}ndig verfasst habe und keine
anderen als die angegebenen Quellen und Hilfsmittel benutzt habe.\par
\vspace{5\baselineskip}
Heidelberg, den (Datum)\hspace{3cm}\dotfill

  \end{appendix}
\end{document}
